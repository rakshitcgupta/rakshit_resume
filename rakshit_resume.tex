\documentclass[11pt]{article} %Sets the default text size to 11pt and class to article.
%------------------------Dimensions--d------------------------------------------
\usepackage{multirow}
\usepackage{tabulary}
\usepackage{hhline}
\usepackage[english]{babel}
\usepackage[utf8x]{inputenc}
\usepackage[protrusion=true,expansion=true]{microtype}
\usepackage{amsmath,amsfonts,amsthm}     % Math packages
\usepackage{graphicx}                    % Enable pdflatex
\usepackage[svgnames]{xcolor}            % Colors by their 'svgnames'
\usepackage{geometry}
	\textheight=700px                    % Saving trees ;-)
\usepackage{url}

\pagestyle{empty}           % No pagenumbers/headers/footers

%%% Custom sectioning (sectsty package)
%%% ------------------------------------------------------------
\usepackage{sectsty}
\sectionfont{%			            % Change font of \section command
	\usefont{OT1}{phv}{b}{n}%		% bch-b-n: CharterBT-Bold font
	\sectionrule{0pt}{0pt}{-3pt}{1pt}}

%%% Macros
%%% ------------------------------------------------------------
\newlength{\spacebox}
\settowidth{\spacebox}{8888888888}			% Box to align text
\newcommand{\sepspace}{\vspace*{1em}}		% Vertical space macro

\newcommand{\MyName}[1]{ % Name
		\Huge \usefont{OT1}{phv}{b}{n} \hfill #1
		\par \normalsize \normalfont}
		
\newcommand{\MySlogan}[1]{ % Slogan (optional)
		\large \usefont{OT1}{phv}{m}{n}\hfill \textit{#1}
		\par \normalsize \normalfont}

\newcommand{\NewPart}[1]{\section*{\uppercase{#1}}}

\newcommand{\PersonalEntry}[2]{
		\noindent\hangindent=2em\hangafter=0 % Indentation
		\parbox{\spacebox}{        % Box to align text
		\textit{#1}}		       % Entry name (birth, address, etc.)
		\hspace{1.5em} #2 \par}    % Entry value

\newcommand{\SkillsEntry}[2]{      % Same as \PersonalEntry
		\noindent\hangindent=2em\hangafter=0 % Indentation
		\parbox{\spacebox}{        % Box to align text
		\textit{#1}}			   % Entry name (birth, address, etc.)
		\hspace{1.5em} #2 \par}    % Entry value	
		
\newcommand{\EducationEntry}[4]{
		\noindent \textbf{#1} \hfill      % Study
		\colorbox{Black}{%
			\parbox{6em}{%
			\hfill\color{White}#2}} \par  % Duration
		\noindent \textit{#3} \par        % School
		\noindent\hangindent=2em\hangafter=0 \small #4 % Description
		\normalsize \par}

\newcommand{\WorkEntry}[4]{				  % Same as \EducationEntry
		\noindent \textbf{#1} \hfill      % Jobname
		\colorbox{Black}{\color{White}#2} \par  % Duration
		\noindent \textit{#3} \par              % Company
		\noindent\hangindent=2em\hangafter=0 \small #4 % Description
		\normalsize \par}

\usepackage{array}
\topmargin=0.0in %length of margin at the top of the page (1 inch added by default)
\oddsidemargin=0.0in %length of margin on sides for odd pages
\evensidemargin=0in %length of margin on sides for even pages
\textwidth=6.5in %How wide you want your text to be
\marginparwidth=0.5in
\headheight=0pt %1in margins at top and bottom (1 inch is added to this value by default)
\headsep=0pt %Increase to increase white space in between headers and the top of the page
\textheight=9.0in %How tall the text body is allowed to be on each page
\begin{document}
%These two pieces of code tell LaTeX that everything that goes in between these tags is what you want displayed as your actual document.


\centerline{{\Huge \sc \textbf {Rakshit Gupta} }}  %Makes whatever text you put in parenthesis move to the center
%Prevents the following text from being indented
%This is the same as a return in Latex
\centerline{\fontsize{6}{6}\selectfont COMPUTER ENGINEERING STUDENT}
\centerline{07219011950 \textbullet \hspace{5pt}rakshitgupta50@gmail.com \textbullet \hspace{5pt} }
\centerline{B-6, BILT Old paper mill colony, Ballarpur, Dist. Chandrapur, Maharashtra}
\noindent



\NewPart{ Academics}{}

\newcolumntype{P}[1]{>{\centering\arraybackslash}p{#1}}
 \begin{tabular}{|P{3cm}|P{5.5cm}|P{3cm}|P{3cm}|}

   \hline
  
    \textbf {Examination} & \textbf { College/School} & \textbf {Passing Year} & \textbf { Percentage/CGPA Obtained} \\ \hline
	
    B.E. Sem-4(MU) & \multirow{4}{5.5cm}{\centering Sardar Patel Institute Of Technology, Mumbai} & June 2017 & 9  \\ 
    \hhline{-~--}
    B.E. Sem-3(MU) &  & Jan 2017 & 8.8  \\ \hhline{-~--}
    B.E. Sem-2(MU) &  & June 2016 & 8.06 \\ \hhline{-~--}
    B.E. Sem-1(MU) &  & Jan 2016 & 7.8  \\ \hline
    \centering Class XII (BIE) & Vijaya Ratna Jr College, Hyderabad  & April 2015  & 97.4\%  \\ \hline
    Class X (CBSE) & Modern English School, Jeypore  & March 2013 & 10  \\  \hline

  \end{tabular}
  \newline






\NewPart{    Technical Skills}{}

\vspace{0pt}

\begin{itemize}
\itemsep-0.5em 
\item \textbf{Programming Languages:} C, Java, Python

\item \textbf{Version Control System:} Github
\vspace{0pt}

\item \textbf{Web:} HTML, CSS, Bootstrap, Javascript
\item \textbf{Creative/Others:} Adobe Photoshop, Latex, Autocad, Adobe Phone Gap

\item \textbf{IDEs:}  Eclipse, Jetbrains IntelliJ IDEA, Jetbrains Clion, Jetbrains Phpstorm, Android Studio, Dev C++, Turbo C++

\item \textbf{Operating System:} Windows, Linux, Android

\end{itemize}



\NewPart{    Projects}{}
\begin{itemize}
\itemsep-0.5em
\item \textbf{GUI For SMS Broadcast: }\hfill {July 2017 - Present}
\\Designed a GUI in python to broadcast a message string to more than 1000 users
%Function coded in python to take 10 digit numbers from a text file/csv/xlsx file format 
\item \textbf{Othello Game Using Swing Framework In Java: }\hfill {July 2016 - October 2016}
\\Othello is a strategy board game for two players, played on an 8×8 uncheckered board, developed as a Second Year Computer Engineering project for the Object Oriented Programming Methodology Course 
\item \textbf{Encryption-Based Program In C: }\hfill {Jan 2016 - March 2016}\\A simple encryption program which encrypted the text using substitution and rail-fence mechanism, developed as a First Year Computer Engineering project for the Structural Programming Approach course
\end{itemize}


\NewPart{    Personal Abilities}{}

\begin{itemize}
\itemsep-0.5em 
\item Excellent logical , analytical and computational skills
\item Strong motivational and leadership skills
\item Ability to work under pressure 
\item Ability to work as individual as well as team 
\end{itemize}


\NewPart{  Position of Responsibilities}{}

\begin{itemize}
\itemsep-0.5em 
\item \textbf{'Creative Head'} of college’s Computer Society Of India (CSI) committee for the academic year 2017-18
\item \textbf{'Head of Programming'} of college’s IOT committee for the academic year 2017-18
\item Organiser, Github workshop for Second and Third year Computer and IT students 
\item Organiser,'No Escape' event, technical festival MATRIX of college,2016-17 
\end{itemize}

%%% Hobbies and interests
%%% ------------------------------------------------------------
\NewPart{ Interests}{}
Basketball, Watching anime, Playing guitar, Discovering new places and cultures

\end{document}